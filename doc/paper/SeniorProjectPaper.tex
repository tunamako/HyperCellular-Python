\documentclass[letterpaper,12pt]{article}
\newcommand{\forceindent}{\leavevmode{\parindent=1em\indent}}

\usepackage{indentfirst, setspace, amssymb,amsmath}
\doublespacing
\begin{document}

\title{CS 480 - Project Paper}
\author{Michael Whalen}
\date{April 18, 2018}
\maketitle
\setlength{\parindent}{4em}

\section*{Introduction}

When I started this project, I really didn't have a good idea of its scope or what exactly I'd encounter along the way. I chose to do this project because I was vaguely familliar with hyperbolic tilings and thought they looked great, and that they'd be a neat playground for cellular automata. I ended up spending the first half of the semester largely just reading textbooks and math journals to learn how to actually construct the geometric objects I was after. Throughout that whole time I was unsure that I'd actually be able to complete the project and even had another project idea as backup. A couple months in, though, everything seemed to finally click and the code started flowing like water. 

I'll begin by describing the gist of the mathematics behind hyperbolic space, then move into the algorithmic aspect of implementing what I needed. Finally, I'll discuss how I implemented cellular automata and the discoveries made as a result.

\section*{Hyperbolic Geometry}

At the most basic level, \textbf{Hyperbolic Geometry} is exactly like the normal Euclidean geometry we're used to, with one twist: with Euclidean geometry, given a line and a point, there is only one other line through that point that is parallel to the given line. In hyperbolic space, we let there be an infinite amount. This one allowance gives way to a ton of interesting behaviors and can be modeled in a multitude of ways. For my purposes, I chose to use what's known as the Poincare model as it has a significant amount of research in it and was relatively simple to learn more about. In essence, the Poincare model is represented by a disk, wherein 'lines' are defined to be arcs of Euclidean circles orthogonal to the disk.

Like Euclidean geometry, in hyperbolic geometry we can examine polygons. Moreover, we can create a \textbf{tiling} of the space using polygons of many kinds, i.e., the space is filled by interlocking, identical shapes. This project concerns itself only with regular polygons, but irregular polygons could be an interesting thing to explore as well. \textbf{Regular polygons} are shapes whose angle measures are uniform, like squares and pentagons in Euclidean space. The real restriction of Euclidean space however is that there are only three tilings of regular polygons: equilateral triangles, squares, and hexagons. Nothing else works as there will either be overlap of polygons or empty space. This is where the value of hyperbolic space comes into play: there exists an infinte number of regular polygons that tile the hyperbolic plane.

\section*{Describing Tilings}

Since there are so many tilings, I needed a simple and programmatic way of referring to individual tilings. This was done using a \textbf{Schl{\"a}fli Symbol}, notated as \textbf{\{p, q\}}, where $p$ denotes the number of sides on the polygon and $q$ denotes the number of polygons adjacent to any given vertex. For example, in Euclidean space we can represent the square tiling as \{4, 4\} since squares (four sides) meet four at each vertex. This symbol is in fact enough to determine whether a given tiling is Euclidean or hyperbolic. With the following formula:
\begin{gather*}
1/p + 1/q = \theta
\end{gather*}
When $\theta = 1/2$, the tiling is Euclidean. When $\theta > 1/2$, it's hyperbolic. Using these pairs of numbers makes it much easier to implement them in software as a two-value representation of a complex concept is simpler to deal with than an alternative.

\section*{Constructing Tilings}

So we know what a tiling is, but how do we make one? It's actually a question with a suprisingly simple solution, yet it took me a couple of months to realize it through extensive reading. We can construct hyperbolic tilings in exactly the same way that we might in Euclidean space: start with a polygon, reflect it about each of its sides, and repeat for these new polygons. The only real difference with hyperbolic space is the method of reflection. Because the sides of the polygon are sections of circles, reflecting a point about a side is the same as performing a Euclidean inversion about the corresponding circle. 

\section*{Software Implementation}

\section*{Cellular Automata}

\section*{Conclusion}

\section*{Completed Points}






\begin{center}
  \begin{tabular}{ | l | r | }
    \hline
    \textbf{Feature} & \textbf{Point Value} \\ \hline
    Hyperbolic tilings of a single polygon & 20 \\ \hline
    Support arbitrarily many polygon types & 20 \\ \hline
    Color in the tilings according to an automaton & 20 \\ \hline
    Animate changes in state & 10 \\ \hline
    Supports multiple sets of rules & 5 \\ \hline
	Supports an arbitrary amount of rulesets & 10 \\ \hline
	Supports more than two states & 5 \\ \hline
	Users can interactively change the state of cell by clicking & 10 \\ \hline
	User selects cell dimensions & 5 \\ \hline
	User inputs rulesets & 5 \\ \hline
	User inputs possible states & 3 \\ \hline
	User selects color of states & 3 \\ \hline
	User selects speed of animation & 1 \\ \hline
	Start/stop animation & 1 \\ \hline
	Step one generation at a time & 1 \\ \hline
	Can save/load states and rules & 5 \\ \hline
	Can zoom in & 5 \\ \hline
	Zoom without losing resolution & 10 \\ \hline
	Can rotate view & 10 \\ \hline
	\textbf{Total:} & \textbf{149} \\ \hline

  \end{tabular}
\end{center}

\begin{center}
  \begin{tabular}{ | l | r | }
    \hline
    \textbf{Point Range} & \textbf{Grade} \\ \hline
    100+ & A \\ \hline
    85 - 99 & B \\ \hline
    70 - 84 & C \\ \hline
    50 - 69 & D \\ \hline
    Below 50 & F \\ \hline

  \end{tabular}
\end{center}
\end{document}