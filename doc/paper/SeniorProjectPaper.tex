\documentclass[letterpaper,12pt]{article}
\begin{document}

\title{CS 480 - Project Paper}
\author{Michael Whalen}
\date{April 18, 2018}
\maketitle


\begin{center}
  \begin{tabular}{ | l | r | }
    \hline
    \textbf{Feature} & \textbf{Point Value} \\ \hline
    Hyperbolic tilings of a single polygon & 20 \\ \hline
    Support arbitrarily many polygon types & 20 \\ \hline
    Color in the tilings according to an automaton & 20 \\ \hline
    Animate changes in state & 10 \\ \hline
    Supports multiple sets of rules & 5 \\ \hline
	Supports an arbitrary amount of rulesets & 10 \\ \hline
	Supports more than two states & 5 \\ \hline
	Users can interactively change the state of cell by clicking & 10 \\ \hline
	User selects cell dimensions & 5 \\ \hline
	User inputs rulesets & 5 \\ \hline
	User inputs possible states & 3 \\ \hline
	User selects color of states & 3 \\ \hline
	User selects speed of animation & 1 \\ \hline
	Start/stop animation & 1 \\ \hline
	Step one generation at a time & 1 \\ \hline
	Can save/load states and rules & 5 \\ \hline
	Can zoom in & 5 \\ \hline
	Zoom without losing resolution & 10 \\ \hline
	Can rotate view & 10 \\ \hline
	\textbf{Total:} & \textbf{149} \\ \hline

  \end{tabular}
\end{center}

\begin{center}
  \begin{tabular}{ | l | r | }
    \hline
    \textbf{Point Range} & \textbf{Grade} \\ \hline
    100+ & A \\ \hline
    85 - 99 & B \\ \hline
    70 - 84 & C \\ \hline
    50 - 69 & D \\ \hline
    Below 50 & F \\ \hline

  \end{tabular}
\end{center}
\end{document}